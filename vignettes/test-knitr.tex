%\VignetteIndexEntry{Creating IGV HTML reports with tracktables}
%\VignettePackage{tracktables}
%\VignetteEngine{knitr::knitr}

% To compile this document
% library('knitr'); rm(list=ls()); knit('tracktables.Rnw')

\documentclass[12pt]{article}




\RequirePackage{/home/pgellert/Downloads/R-3.1.0/library/BiocStyle/sty/Bioconductor}

\AtBeginDocument{\bibliographystyle{/home/pgellert/Downloads/R-3.1.0/library/BiocStyle/sty/unsrturl}}





\author{Thomas Carroll$^{1*}$\\[1em] \small{$^{1}$ Bioinformatics Facility, MRC Clincal Sciences Centre;} \\ \small{\texttt{$^*$thomas.carroll (at)imperial.ac.uk}}}

\title{Creating IGV HTML reports with tracktables}

\begin{document}




\maketitle

\begin{abstract}
 
The visualisation of genomics data in genome browsers is a key step in both quality control and initial interrogation of hypothesis under investigation. 

The organisation of large collections of genomics data (such as large scale High-thoughput sequencing data experiments) alongside assocated Sample or Experimental metadata allows for rapid evaluation of patterns across such experimental groups but such organisation maybe very time consuming.

 The Tracktables package provides a set of tools to build IGV session files from sample files and associated metadata as well as produce IGV linked HTML reports for high thoroughput visualisation of sample data in IGV.
 
  \vspace{1em}
  
  \end{abstract}



\newpage

\tableofcontents

\section{Creating IGV sessions and HTML reports using tracktables}

The three main functions within tracktables package are the MakeIGVSessionXML() and  MakeIGVSampleMetadata() functions for a creating IGV session for sample files and sample metadata and the maketracktable() function to create HTML pages with tables to control IGV alongside sample information and metadata,

\subsection{Creating input files for tracktables}

Tracktables functions require the user to provide both a matrix or dataframe of metadata information and one of sample file locations to be visualised in IGV. 

These matricies/data-frames must both have one column names "SampleSheet" which contains unique sample IDs.

The remaining metadata samplesheet columns may be user-defined but must all contain column titles. (See example below)
The sample filesheet must contain the columns "SampleName", "bam","bigwig" and interval. These columns may contain NA values when no relavant file is associated to a sample.

Here we create a small example SampleSheet (containing metadata) and FileSheet (containing file locations)












